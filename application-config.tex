%% -----------------------------------------------------
%% Configuration of an NSERC application
%% Defines a few macros that are global to all documents of the application
%% ----------------------------------------------------

%% If your application involves a company, put the name of the company
%% in a macro rather than writing it directly.
\newcommand{\namecompany}{Temporal inc.}

%% Application year. Only appears in the metadata of the generated PDF.
\newcommand{\applicationyear}{2014}

%% The list of all authors of the application. Again, only useful for the
%% PDF metadata
\newcommand{\authorlist}{Emmett Brown}

%% The name and NSERC PIN of the main applicant
\nsercname{Emmett Brown}
\nsercpin{999999}

%% Documents are not dated
\date{}

%% Paragraphes français
%\setlength{\parindent}{0pt}

%% Hack to have list items displayed in a more compact way
\usepackage{paralist}
\setlength{\pltopsep}{4pt}
\setlength{\plitemsep}{4pt}

%% ----------
%% Loading a few packages. These are all optional and can be commented
%% out if you with. Feel free to add others.
%% ----------
\usepackage{hyperref}
\hypersetup{%
  pdfauthor = {\authorlist{}},
  pdfsubject = {NSERC \applicationyear{} \namecompany{}}
}
\usepackage{microtype}
\usepackage{url}
\usepackage[x11names]{xcolor}
\usepackage[]{todonotes}
\usepackage{comment}

%% Useful: a few "todo" macros to display colored boxes with remarks
%% and comments
\newcommand{\todoemmett}[1]{\todo[inline,color=OliveDrab1]{\sf\small Emmett: #1}}
\newcommand{\todomarty}[1]{\todo[inline,color=LightBlue2]{\sf\small Marty: #1}}
\newcommand{\todoall}[1]{\todo[inline,color=yellow]{\sf\small #1}}
